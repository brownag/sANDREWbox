\documentclass[]{article}
\usepackage{lmodern}
\usepackage{amssymb,amsmath}
\usepackage{ifxetex,ifluatex}
\usepackage{fixltx2e} % provides \textsubscript
\ifnum 0\ifxetex 1\fi\ifluatex 1\fi=0 % if pdftex
  \usepackage[T1]{fontenc}
  \usepackage[utf8]{inputenc}
\else % if luatex or xelatex
  \ifxetex
    \usepackage{mathspec}
  \else
    \usepackage{fontspec}
  \fi
  \defaultfontfeatures{Ligatures=TeX,Scale=MatchLowercase}
\fi
% use upquote if available, for straight quotes in verbatim environments
\IfFileExists{upquote.sty}{\usepackage{upquote}}{}
% use microtype if available
\IfFileExists{microtype.sty}{%
\usepackage{microtype}
\UseMicrotypeSet[protrusion]{basicmath} % disable protrusion for tt fonts
}{}
\usepackage[margin=1in]{geometry}
\usepackage{hyperref}
\hypersetup{unicode=true,
            pdftitle={Loafercreek - a pedon summary function demonstration},
            pdfauthor={Andrew Brown},
            pdfborder={0 0 0},
            breaklinks=true}
\urlstyle{same}  % don't use monospace font for urls
\usepackage{color}
\usepackage{fancyvrb}
\newcommand{\VerbBar}{|}
\newcommand{\VERB}{\Verb[commandchars=\\\{\}]}
\DefineVerbatimEnvironment{Highlighting}{Verbatim}{commandchars=\\\{\}}
% Add ',fontsize=\small' for more characters per line
\usepackage{framed}
\definecolor{shadecolor}{RGB}{248,248,248}
\newenvironment{Shaded}{\begin{snugshade}}{\end{snugshade}}
\newcommand{\KeywordTok}[1]{\textcolor[rgb]{0.13,0.29,0.53}{\textbf{#1}}}
\newcommand{\DataTypeTok}[1]{\textcolor[rgb]{0.13,0.29,0.53}{#1}}
\newcommand{\DecValTok}[1]{\textcolor[rgb]{0.00,0.00,0.81}{#1}}
\newcommand{\BaseNTok}[1]{\textcolor[rgb]{0.00,0.00,0.81}{#1}}
\newcommand{\FloatTok}[1]{\textcolor[rgb]{0.00,0.00,0.81}{#1}}
\newcommand{\ConstantTok}[1]{\textcolor[rgb]{0.00,0.00,0.00}{#1}}
\newcommand{\CharTok}[1]{\textcolor[rgb]{0.31,0.60,0.02}{#1}}
\newcommand{\SpecialCharTok}[1]{\textcolor[rgb]{0.00,0.00,0.00}{#1}}
\newcommand{\StringTok}[1]{\textcolor[rgb]{0.31,0.60,0.02}{#1}}
\newcommand{\VerbatimStringTok}[1]{\textcolor[rgb]{0.31,0.60,0.02}{#1}}
\newcommand{\SpecialStringTok}[1]{\textcolor[rgb]{0.31,0.60,0.02}{#1}}
\newcommand{\ImportTok}[1]{#1}
\newcommand{\CommentTok}[1]{\textcolor[rgb]{0.56,0.35,0.01}{\textit{#1}}}
\newcommand{\DocumentationTok}[1]{\textcolor[rgb]{0.56,0.35,0.01}{\textbf{\textit{#1}}}}
\newcommand{\AnnotationTok}[1]{\textcolor[rgb]{0.56,0.35,0.01}{\textbf{\textit{#1}}}}
\newcommand{\CommentVarTok}[1]{\textcolor[rgb]{0.56,0.35,0.01}{\textbf{\textit{#1}}}}
\newcommand{\OtherTok}[1]{\textcolor[rgb]{0.56,0.35,0.01}{#1}}
\newcommand{\FunctionTok}[1]{\textcolor[rgb]{0.00,0.00,0.00}{#1}}
\newcommand{\VariableTok}[1]{\textcolor[rgb]{0.00,0.00,0.00}{#1}}
\newcommand{\ControlFlowTok}[1]{\textcolor[rgb]{0.13,0.29,0.53}{\textbf{#1}}}
\newcommand{\OperatorTok}[1]{\textcolor[rgb]{0.81,0.36,0.00}{\textbf{#1}}}
\newcommand{\BuiltInTok}[1]{#1}
\newcommand{\ExtensionTok}[1]{#1}
\newcommand{\PreprocessorTok}[1]{\textcolor[rgb]{0.56,0.35,0.01}{\textit{#1}}}
\newcommand{\AttributeTok}[1]{\textcolor[rgb]{0.77,0.63,0.00}{#1}}
\newcommand{\RegionMarkerTok}[1]{#1}
\newcommand{\InformationTok}[1]{\textcolor[rgb]{0.56,0.35,0.01}{\textbf{\textit{#1}}}}
\newcommand{\WarningTok}[1]{\textcolor[rgb]{0.56,0.35,0.01}{\textbf{\textit{#1}}}}
\newcommand{\AlertTok}[1]{\textcolor[rgb]{0.94,0.16,0.16}{#1}}
\newcommand{\ErrorTok}[1]{\textcolor[rgb]{0.64,0.00,0.00}{\textbf{#1}}}
\newcommand{\NormalTok}[1]{#1}
\usepackage{graphicx,grffile}
\makeatletter
\def\maxwidth{\ifdim\Gin@nat@width>\linewidth\linewidth\else\Gin@nat@width\fi}
\def\maxheight{\ifdim\Gin@nat@height>\textheight\textheight\else\Gin@nat@height\fi}
\makeatother
% Scale images if necessary, so that they will not overflow the page
% margins by default, and it is still possible to overwrite the defaults
% using explicit options in \includegraphics[width, height, ...]{}
\setkeys{Gin}{width=\maxwidth,height=\maxheight,keepaspectratio}
\IfFileExists{parskip.sty}{%
\usepackage{parskip}
}{% else
\setlength{\parindent}{0pt}
\setlength{\parskip}{6pt plus 2pt minus 1pt}
}
\setlength{\emergencystretch}{3em}  % prevent overfull lines
\providecommand{\tightlist}{%
  \setlength{\itemsep}{0pt}\setlength{\parskip}{0pt}}
\setcounter{secnumdepth}{0}
% Redefines (sub)paragraphs to behave more like sections
\ifx\paragraph\undefined\else
\let\oldparagraph\paragraph
\renewcommand{\paragraph}[1]{\oldparagraph{#1}\mbox{}}
\fi
\ifx\subparagraph\undefined\else
\let\oldsubparagraph\subparagraph
\renewcommand{\subparagraph}[1]{\oldsubparagraph{#1}\mbox{}}
\fi

%%% Use protect on footnotes to avoid problems with footnotes in titles
\let\rmarkdownfootnote\footnote%
\def\footnote{\protect\rmarkdownfootnote}

%%% Change title format to be more compact
\usepackage{titling}

% Create subtitle command for use in maketitle
\newcommand{\subtitle}[1]{
  \posttitle{
    \begin{center}\large#1\end{center}
    }
}

\setlength{\droptitle}{-2em}

  \title{Loafercreek - a pedon summary function demonstration}
    \pretitle{\vspace{\droptitle}\centering\huge}
  \posttitle{\par}
    \author{Andrew Brown}
    \preauthor{\centering\large\emph}
  \postauthor{\par}
    \date{}
    \predate{}\postdate{}
  

\begin{document}
\maketitle

The Loafercreek series (fine-loamy, mixed, active, thermic Ultic
Haploxeralfs) is comprised of soils on foothills underlain by
metavolcanic rock. They are moderately deep (50 to 100cm) to a
paralithic contact. The series was established in Butte county (CA612)
and now is mapped in Calaveras and Tuolumne (CA630) as well as small
portions of Mariposa and Stanislaus Counties (CA630 join with CA649 and
CA644).

Areas of soils similar to Loafercreek have been mapped as Auburn soils
throughout the Sierra Nevada Foothills. These Auburn soils no longer
fall within a single family (12th edition taxonomy). Auburn ``series''
has been constrained to shallow soils, specifically the lithic subgroups
lacking a clay increase (Lithic Haploxerepts).

Deeper soils on these parent materials and in this climate typically
have a pedogenic clay increase (argillic horizon). Shallow soils with
argillic horizons are the geographically associated Bonanza and Dunstone
soils. Skeletal varieties are Jasperpeak (shallow) and Gopherridge
(moderately deep). Soils without a bedrock restriction are called
Motherlode.

In this demo, we are going to use R-based summaries to explore some of
the properties of the Loafercreek soils found during soil survey
inventory in CA630. An understanding of the range in characteristics of
these soils will facilitate correlation of modern soil series concepts
during FY19 update work on the adjacent CA649 (Mariposa County Area)
which contains large areas of similar metavolcanic rocks.

\subsubsection{Load the data}\label{load-the-data}

The \texttt{sharpshootR} library has some great tools for manipulating
pedon data. To get the data out of the complex structures found in
databases and into a (still complex) R-usable format, we use the library
\texttt{soilDB}. One of the default test datasets provided by
\texttt{soilDB} is \texttt{loafercreek}. Along with \texttt{soilDB} we
get the dependency \texttt{aqp} which gives us the data structure to
hold our pedon data (a \texttt{SoilProfileCollection} object).

\begin{Shaded}
\begin{Highlighting}[]
\KeywordTok{library}\NormalTok{(sharpshootR)}
\KeywordTok{library}\NormalTok{(soilDB)}
\end{Highlighting}
\end{Shaded}

\begin{verbatim}
## Loading required package: aqp
\end{verbatim}

\begin{verbatim}
## This is aqp 1.16-5
\end{verbatim}

\begin{Shaded}
\begin{Highlighting}[]
\KeywordTok{data}\NormalTok{(}\StringTok{"loafercreek"}\NormalTok{)}
\end{Highlighting}
\end{Shaded}

\subsubsection{The
SoilProfileCollection}\label{the-soilprofilecollection}

The \texttt{loafercreek} dataset we just loaded is a
\texttt{SoilProfileCollection} (SPC) object. A SPC is a multipart (S4) R
object that contains paired site and horizon-level data for soil profile
(pedon) observations. It can contain spatial data in the spatial slot
(\texttt{spc@sp}). The site (\texttt{spc@site}) and horizon
(\texttt{spc@horizon}) slots are each comprised of a single
\texttt{data.frame}. They should be accessed using
\texttt{horizons(spc)} and \texttt{site(spc)} or using square bracket
notation.

\begin{quote}
Read more about the \texttt{SoilProfileCollection} object
\end{quote}

Let's check out \texttt{loafercreek}!

\begin{Shaded}
\begin{Highlighting}[]
\CommentTok{# plot a few profiles}
\KeywordTok{plot}\NormalTok{(}\KeywordTok{head}\NormalTok{(loafercreek))}
\end{Highlighting}
\end{Shaded}

\begin{verbatim}
## guessing horizon designations are stored in `hzname`
\end{verbatim}

\includegraphics{loafercreek_20181221_files/figure-latex/unnamed-chunk-1-1.pdf}

\begin{Shaded}
\begin{Highlighting}[]
\CommentTok{#number of sites (profiles) in `loafercreek` site dataframe}
\KeywordTok{nrow}\NormalTok{(}\KeywordTok{site}\NormalTok{(loafercreek))}
\end{Highlighting}
\end{Shaded}

\begin{verbatim}
## [1] 115
\end{verbatim}

\begin{Shaded}
\begin{Highlighting}[]
\CommentTok{#number of horizons (layers) in `loafercreek` horizons dataframe}
\KeywordTok{nrow}\NormalTok{(}\KeywordTok{horizons}\NormalTok{(loafercreek))}
\end{Highlighting}
\end{Shaded}

\begin{verbatim}
## [1] 678
\end{verbatim}

\begin{Shaded}
\begin{Highlighting}[]
\CommentTok{#maximum clay content observed out of all profiles and horizons (max of `clay` variable in the horizons dataframe)}
\KeywordTok{max}\NormalTok{(}\KeywordTok{horizons}\NormalTok{(loafercreek)}\OperatorTok{$}\NormalTok{clay, }\DataTypeTok{na.rm =} \OtherTok{TRUE}\NormalTok{)}
\end{Highlighting}
\end{Shaded}

\begin{verbatim}
## [1] 60
\end{verbatim}

There is a many:one relationship for horizons within a site. Many of the
properties we describe in soil survey (geomorphology, color, structure,
rock fragments etc) also are many:one with respect to site or horizon.
The SPC data structure (linking site and horizon data) provides a way of
keeping this all aligned, or at least it attempts to. :)

Many-to-one ``flattening'' used in SPCs created from hierarchical
databases often makes use of RV indicator fields, or other compression
of information within child tables, to provide a single site or horizon-
level value as appropriate. If data are populated incorrectly
(particularly in NASIS applications) this may lead to duplication of
records.

Sometimes you want to do your own flattening. Many analyses start with
creating site-level variables based on an aggregation of some of a
pedon's horizon data; and you would like to apply the calculation to
each profile in your dataset. For instance, you might be interested in
calculating weighted-average field-texture clay content in the particle
size control section, or the fine-clay ratio in the first ``illuvial''
horizon, or the whole-profile available water capacity.

SPC data (sites, horizons) are accessed in ways similar to a base R
\texttt{data.frame}. Despite the similarities, remember that most R
functions are not ``aware'' of SPCs and will often fail in unsual ways
when supplied with one as an argument.

\texttt{soilDB} provides a variety of ways to import soil data into
SPCs, or you can create your own from scratch/flat files. \texttt{aqp}
and \texttt{sharpshootR} provide a wide array of functions (and more all
the time) for interacting with SPCs.

\begin{center}\rule{0.5\linewidth}{\linethickness}\end{center}

\subsubsection{\texorpdfstring{Using
\texttt{profileApply()}}{Using profileApply()}}\label{using-profileapply}

The \texttt{aqp} function \texttt{profileApply()} allows us to evaluate
a user-specified function on each profile in a SPC.
\texttt{profileApply()} returns a \texttt{list} object containing
results of each function call. The number of results returned is equal
to the number of sites in the SPC.

Here, we use a \texttt{sharpshootR} function called
\texttt{estimateSoilDepth()} to demonstrate how \texttt{profileApply()}
works.

\begin{quote}
Read more about \texttt{estimateSoilDepth()}
\end{quote}

\texttt{estimateSoilDepth()} use a REGular EXpression (regex pattern) to
match horizon designations. The default setting is designed for bedrock
restrictions and matches horizon designations matching Cr, R or Cd. The
function returns the top depth of the first horizon matching that
pattern.

\begin{Shaded}
\begin{Highlighting}[]
\CommentTok{#call the function 'estimateSoilDepth()' on each profile in SPC 'loafercreek'}
\NormalTok{depth.to.contact <-}\StringTok{ }\KeywordTok{profileApply}\NormalTok{(loafercreek, estimateSoilDepth)}
\end{Highlighting}
\end{Shaded}

It is possible, depending on the user-specified function ``applied,'' to
have results that are \texttt{NA} or behave unexpectedly (i.e.~results
with length \textgreater{} 1, complex objects). Before writing a
\texttt{profileApply()} call, try your function on just a single
profile:

\begin{Shaded}
\begin{Highlighting}[]
\NormalTok{just.one <-}\StringTok{ }\NormalTok{loafercreek[}\DecValTok{1}\NormalTok{]}
\KeywordTok{estimateSoilDepth}\NormalTok{(just.one)}
\end{Highlighting}
\end{Shaded}

\begin{verbatim}
## [1] 79
\end{verbatim}

Do this to make sure you know what your output will look like for the
whole collection (it will be a \texttt{list}). Sometimes a function will
fail for only certain pedons. This may reflect a data population error
in that pedon or it may be a problem with the function's algorithm not
accounting for the unique case of that pedon.

This function is far more versatile than may first be apparent. By
changing the default settings you can calculate other ``depth-to-X''
type properties. You could add duripan or petrocalcic if that was
relevant to your data. It doesn't have to be a `root-restriction' depth,
either.

Here are some (potentially) noncemented exampless: + depth to carbonates
- match horizon with k or kk subscript -
\texttt{estimateSoilDepth(pedon,\ p\ =\ \textquotesingle{}k?k\textquotesingle{})}
+ depth to gley (\textasciitilde{}2 chroma dominant color) -
\texttt{estimateSoilDepth(pedon,\ p\ =\ \textquotesingle{}g\textquotesingle{})}

\begin{quote}
\texttt{estimateSoilDepth()} is a generic regex pattern matcher for any
text-based / character field in a horizon specified by \texttt{name}.
Using moist hue instead of horizon designation, you can, for example,
estimate depth to 2.5YR moist hue:
\texttt{estimateSoilDepth(pedon,\ name=\textquotesingle{}m\_hue\textquotesingle{},\ p=\textquotesingle{}\^{}2.5YR\textquotesingle{})}
\end{quote}

\href{https://www.regular-expressions.info/}{More about regex}

Lets see how our depth to contact data look.

\begin{Shaded}
\begin{Highlighting}[]
\CommentTok{#look at a density (frequency) plot; depth on x axis}
\KeywordTok{plot}\NormalTok{(}\KeywordTok{density}\NormalTok{(depth.to.contact, }\DataTypeTok{na.rm =} \OtherTok{TRUE}\NormalTok{))}
\end{Highlighting}
\end{Shaded}

\includegraphics{loafercreek_20181221_files/figure-latex/unnamed-chunk-5-1.pdf}

\begin{Shaded}
\begin{Highlighting}[]
\KeywordTok{quantile}\NormalTok{(depth.to.contact, }\DataTypeTok{probs=}\KeywordTok{c}\NormalTok{(}\DecValTok{0}\NormalTok{,}\FloatTok{0.01}\NormalTok{,}\FloatTok{0.05}\NormalTok{,}\FloatTok{0.25}\NormalTok{,}\FloatTok{0.5}\NormalTok{,}\FloatTok{0.75}\NormalTok{,}\FloatTok{0.95}\NormalTok{,}\FloatTok{0.99}\NormalTok{,}\DecValTok{1}\NormalTok{), }\DataTypeTok{na.rm =} \OtherTok{TRUE}\NormalTok{)}
\end{Highlighting}
\end{Shaded}

\begin{verbatim}
##     0%     1%     5%    25%    50%    75%    95%    99%   100% 
##  40.00  45.42  52.00  59.00  66.00  76.00  92.00  98.72 148.00
\end{verbatim}

Mostly moderately deep (50-100cm).

We will compare the calculated values to those ``populated'' (i.e.~come
with \texttt{loafercreek} dataset). If they match they will plot on the
1:1 line. Based on the below result, we will replace the prior values
with the new calculated values.

\begin{Shaded}
\begin{Highlighting}[]
\CommentTok{#plot difference "populated v.s. calculated"}
\KeywordTok{plot}\NormalTok{(loafercreek}\OperatorTok{$}\NormalTok{bedrckdepth }\OperatorTok{~}\StringTok{ }\NormalTok{depth.to.contact, }\DataTypeTok{xlim=}\KeywordTok{c}\NormalTok{(}\DecValTok{0}\NormalTok{,}\DecValTok{200}\NormalTok{), }\DataTypeTok{ylim=}\KeywordTok{c}\NormalTok{(}\DecValTok{0}\NormalTok{,}\DecValTok{200}\NormalTok{))}
\KeywordTok{abline}\NormalTok{(}\DecValTok{0}\NormalTok{, }\DecValTok{1}\NormalTok{)}
\end{Highlighting}
\end{Shaded}

\includegraphics{loafercreek_20181221_files/figure-latex/unnamed-chunk-6-1.pdf}

\begin{Shaded}
\begin{Highlighting}[]
\CommentTok{# some plot off the 1:1 line, lets take the calculated}
\CommentTok{#overwrite populated with calculated depths (based on the po)}
\NormalTok{loafercreek}\OperatorTok{$}\NormalTok{bedrckdepth <-}\StringTok{ }\NormalTok{depth.to.contact}
\end{Highlighting}
\end{Shaded}

You can also define your own functions for use with
\texttt{profileApply()}.

This function defined below calculates the profile maximum clay content.
It applies the function \texttt{max} to the attribute \texttt{clay} in a
user-specified profile \texttt{p}. When called via profileApply,
\texttt{p} is an individual pedon as the apply function iterates over
the \texttt{loafercreek} profiles one-by-one.

\begin{Shaded}
\begin{Highlighting}[]
\NormalTok{profileMaxClay <-}\StringTok{ }\ControlFlowTok{function}\NormalTok{(p, ...) \{}
  \KeywordTok{return}\NormalTok{(}\KeywordTok{profile.univar.func}\NormalTok{(p, }\DataTypeTok{attr =} \StringTok{'clay'}\NormalTok{, }\DataTypeTok{fun=}\NormalTok{max))}
\NormalTok{\}}
\end{Highlighting}
\end{Shaded}

Apply \texttt{profileMaxClay()} to \texttt{loafercreek}

\begin{Shaded}
\begin{Highlighting}[]
\NormalTok{loafercreek}\OperatorTok{$}\NormalTok{maxclay <-}\StringTok{ }\KeywordTok{profileApply}\NormalTok{(loafercreek, profileMaxClay)}
\KeywordTok{plot}\NormalTok{(}\KeywordTok{density}\NormalTok{(loafercreek}\OperatorTok{$}\NormalTok{maxclay, }\DataTypeTok{na.rm=}\NormalTok{T))}
\end{Highlighting}
\end{Shaded}

\includegraphics{loafercreek_20181221_files/figure-latex/unnamed-chunk-9-1.pdf}

Here is the underlying generic function used for accessing a horizon
level variable, performing a function using that variable as input, and
returning the result. The generic is written so that the same underlying
machinery can be used to create many convenience summary functions for
multiple attributes.

\begin{itemize}
\tightlist
\item
  These are the arguments:
\item
  \texttt{spc} - an SPC; which generally should contain a single profile
  (not enforced)
\item
  \texttt{attr} - a column name in \texttt{horizons(spc)}; e.g. `clay'
\item
  \texttt{fun} - an R function to be applied to \texttt{attr} in
  \texttt{spc}; e.g. \texttt{max} to compute the maximum value of a
  numeric vector
\item
  \texttt{na.rm} - default is \texttt{TRUE}; to remove NA values from
  the list of inputs to \texttt{fun}
\item
  \texttt{...} - any subsequent arguments are passed to \texttt{fun} (in
  addition to the \texttt{horizons(spc){[}{[}attr{]}{]}} data)
\end{itemize}

\begin{Shaded}
\begin{Highlighting}[]
\NormalTok{profile.univar.func <-}\StringTok{ }\ControlFlowTok{function}\NormalTok{(spc, attr, fun, }\DataTypeTok{na.rm =} \OtherTok{TRUE}\NormalTok{, ...) \{}
  \CommentTok{#'attr' contains a column name in horizons data frame}
\NormalTok{  d <-}\StringTok{ }\KeywordTok{horizons}\NormalTok{(spc)[[attr]]}
  \CommentTok{#remove NAs}
  \ControlFlowTok{if}\NormalTok{(na.rm)}
\NormalTok{    d <-}\StringTok{ }\NormalTok{d[}\OperatorTok{!}\KeywordTok{is.na}\NormalTok{(d)]}
  \CommentTok{#if input data empty after removing NA, return NA without calling function}
  \CommentTok{#TODO: warning? pedonID list via profileApply()?}
  \ControlFlowTok{if}\NormalTok{(}\OperatorTok{!}\KeywordTok{length}\NormalTok{(d)) }
    \KeywordTok{return}\NormalTok{(}\OtherTok{NA}\NormalTok{)}
  \CommentTok{#pass the data to 'fun' as a list along with any other arguments}
\NormalTok{  rez <-}\StringTok{ }\KeywordTok{unlist}\NormalTok{(}\KeywordTok{do.call}\NormalTok{(fun, }\KeywordTok{list}\NormalTok{(d), ...))}
  \CommentTok{#return the result}
  \KeywordTok{return}\NormalTok{(rez)}
\NormalTok{\}}
\end{Highlighting}
\end{Shaded}

There are other generic functions for more complex behavior (involving
multiple attributes, depth weighted averaging, application of taxonomic
criteria etc.).

\subsubsection{Application of pedon summary
functions}\label{application-of-pedon-summary-functions}

We will see if two common indicators of landscape stability and
pedogenic development (clay content, and redness) are related within the
soils correlated to Loafercreek. Generally, these soils are quite red.
But in areas with thicker colluvial material on the surface or due to
variation within the parent rock, the colors range duller and browner.
In this setting, a soil that is red close to the surface is likely to
stay quite red with depth. We will therefore calculate the depth to 5YR
(or 2.5YR) dry hue as a surrogate for profile ``redness'' with the
expectation that this will capture our reddest soils.

\begin{Shaded}
\begin{Highlighting}[]
\NormalTok{loafercreek}\OperatorTok{$}\NormalTok{depth.to.5YR <-}\StringTok{ }\KeywordTok{profileApply}\NormalTok{(loafercreek, estimateSoilDepth, }\DataTypeTok{name=}\StringTok{'d_hue'}\NormalTok{, }\DataTypeTok{p=}\StringTok{'^5YR|^2.5YR'}\NormalTok{, }\DataTypeTok{no.contact.depth=}\DecValTok{150}\NormalTok{, }\DataTypeTok{no.contact.assigned=}\OtherTok{NA}\NormalTok{)}
\KeywordTok{plot}\NormalTok{(}\DataTypeTok{main=}\StringTok{"Distribution of depth to 5YR or 2.5YR dry hue in Loafercreek"}\NormalTok{, }
     \KeywordTok{density}\NormalTok{(loafercreek}\OperatorTok{$}\NormalTok{depth.to.5YR, }\DataTypeTok{na.rm =}\NormalTok{ T, }\DataTypeTok{from =} \DecValTok{0}\NormalTok{, }\DataTypeTok{to =} \DecValTok{100}\NormalTok{))}
\end{Highlighting}
\end{Shaded}

\includegraphics{loafercreek_20181221_files/figure-latex/unnamed-chunk-11-1.pdf}

Interesting. There might be two clusters here? Or just noise? Plot
profile plots.

\begin{Shaded}
\begin{Highlighting}[]
\NormalTok{sub1 <-}\StringTok{ }\NormalTok{(}\KeywordTok{subsetProfiles}\NormalTok{(loafercreek, }\DataTypeTok{s =} \StringTok{'depth.to.5YR > 40'}\NormalTok{))}
\KeywordTok{plot}\NormalTok{(sub1)}
\end{Highlighting}
\end{Shaded}

\begin{verbatim}
## guessing horizon designations are stored in `hzname`
\end{verbatim}

\includegraphics{loafercreek_20181221_files/figure-latex/unnamed-chunk-12-1.pdf}

\begin{Shaded}
\begin{Highlighting}[]
\NormalTok{sub2 <-}\StringTok{ }\NormalTok{(}\KeywordTok{subsetProfiles}\NormalTok{(loafercreek, }\DataTypeTok{s =} \StringTok{'depth.to.5YR <= 40'}\NormalTok{))}
\KeywordTok{plot}\NormalTok{(sub2)}
\end{Highlighting}
\end{Shaded}

\begin{verbatim}
## guessing horizon designations are stored in `hzname`
\end{verbatim}

\includegraphics{loafercreek_20181221_files/figure-latex/unnamed-chunk-12-2.pdf}

\begin{Shaded}
\begin{Highlighting}[]
\KeywordTok{aggregateColorPlot}\NormalTok{(}\KeywordTok{aggregateColor}\NormalTok{(loafercreek))}
\end{Highlighting}
\end{Shaded}

\includegraphics{loafercreek_20181221_files/figure-latex/unnamed-chunk-13-1.pdf}

\begin{Shaded}
\begin{Highlighting}[]
\KeywordTok{aggregateColorPlot}\NormalTok{(}\KeywordTok{aggregateColor}\NormalTok{(sub1))}
\end{Highlighting}
\end{Shaded}

\includegraphics{loafercreek_20181221_files/figure-latex/unnamed-chunk-13-2.pdf}

\begin{Shaded}
\begin{Highlighting}[]
\KeywordTok{aggregateColorPlot}\NormalTok{(}\KeywordTok{aggregateColor}\NormalTok{(sub2))}
\end{Highlighting}
\end{Shaded}

\includegraphics{loafercreek_20181221_files/figure-latex/unnamed-chunk-13-3.pdf}
In the full dataset, 5YR is the second most common hue after 7.5YR. 10YR
hues basically only occur in the surface horizons or in lower gradtional
horizons/bedrock.

Both the full dataset and the first subset (greater than 40cm to 5YR or
redder) have 7.5YR hue in the upper argillic (Bt1).

In the subset of pedons less than 40cm to 5YR hues, the only 7.5YR chip
in the Bt1 ius 7.5YR 4/6 (i.e.~on higher chroma side).

So, it looks like there might be some decent separation on redness by
separating the two clusters we observed in the density plot. It may be
hard to see with so many profiles. The ones shallower to 5YR are redder
/ higher chroma overall. Lets see if that lines up with our
\texttt{maxclay} we just calculated\ldots{}

\begin{Shaded}
\begin{Highlighting}[]
\KeywordTok{quantile}\NormalTok{(sub1}\OperatorTok{$}\NormalTok{maxclay, }\DataTypeTok{na.rm =}\NormalTok{ T, }\DataTypeTok{probs=}\KeywordTok{c}\NormalTok{(}\DecValTok{0}\NormalTok{,}\FloatTok{0.05}\NormalTok{,}\FloatTok{0.25}\NormalTok{,}\FloatTok{0.5}\NormalTok{,}\FloatTok{0.75}\NormalTok{,}\FloatTok{0.95}\NormalTok{,}\DecValTok{1}\NormalTok{))}
\end{Highlighting}
\end{Shaded}

\begin{verbatim}
##   0%   5%  25%  50%  75%  95% 100% 
## 18.0 21.0 26.0 28.0 32.0 35.8 44.0
\end{verbatim}

\begin{Shaded}
\begin{Highlighting}[]
\KeywordTok{quantile}\NormalTok{(sub2}\OperatorTok{$}\NormalTok{maxclay, }\DataTypeTok{na.rm =}\NormalTok{ T, }\DataTypeTok{probs=}\KeywordTok{c}\NormalTok{(}\DecValTok{0}\NormalTok{,}\FloatTok{0.05}\NormalTok{,}\FloatTok{0.25}\NormalTok{,}\FloatTok{0.5}\NormalTok{,}\FloatTok{0.75}\NormalTok{,}\FloatTok{0.95}\NormalTok{,}\DecValTok{1}\NormalTok{))}
\end{Highlighting}
\end{Shaded}

\begin{verbatim}
##   0%   5%  25%  50%  75%  95% 100% 
## 18.0 20.5 26.2 34.0 38.5 50.7 60.0
\end{verbatim}

While the profiles have similar textures at the lower percentiles
(surface horizons are similar texture), their medians differ by 6\% and
their 95th percentiles by nearly 15\%. The soils with red colors closer
to the surface have the higher max clay contents. That said, with these
ranges, most of these pedons probably fall within the fine-loamy
particle size family range and may just represent two end members within
the series concept.

\begin{Shaded}
\begin{Highlighting}[]
\KeywordTok{library}\NormalTok{(lattice)}
\KeywordTok{library}\NormalTok{(grid)}
\CommentTok{#are these all fine loamy? slabwise; then PSCS estimation}
\CommentTok{#make grouping var}
\NormalTok{loafercreek}\OperatorTok{$}\NormalTok{red.shallow <-}\StringTok{ }\NormalTok{loafercreek}\OperatorTok{$}\NormalTok{depth.to.5YR }\OperatorTok{<=}\StringTok{ }\DecValTok{40}

\CommentTok{#calculate slabs}
\NormalTok{loaf.slab <-}\StringTok{ }\KeywordTok{slab}\NormalTok{(loafercreek[}\OperatorTok{!}\KeywordTok{is.na}\NormalTok{(loafercreek}\OperatorTok{$}\NormalTok{red.shallow)], red.shallow }\OperatorTok{~}\StringTok{ }\NormalTok{clay)}

\CommentTok{#(excluding slabs with contributing fraction < 15%)}
\NormalTok{loaf.slab2 <-}\StringTok{ }\NormalTok{loaf.slab[loaf.slab}\OperatorTok{$}\NormalTok{contributing_fraction }\OperatorTok{>}\StringTok{ }\FloatTok{0.15}\NormalTok{,]}

\CommentTok{#make lattice plot}
\KeywordTok{xyplot}\NormalTok{(top }\OperatorTok{~}\StringTok{ }\NormalTok{p.q50 }\OperatorTok{|}\StringTok{ }\NormalTok{red.shallow, }\DataTypeTok{data=}\NormalTok{loaf.slab2, }
       \DataTypeTok{main=}\StringTok{'Loafercreek - "shallow to 5YR hue" clay}\CharTok{\textbackslash{}n}\StringTok{ distribution with depth'}\NormalTok{,}
       \DataTypeTok{ylab=}\StringTok{'Depth'}\NormalTok{,}
             \DataTypeTok{xlab=}\StringTok{'median bounded by 25th and 75th percentiles'}\NormalTok{,}
             \DataTypeTok{lower=}\NormalTok{loaf.slab2}\OperatorTok{$}\NormalTok{p.q25, }\DataTypeTok{upper=}\NormalTok{loaf.slab2}\OperatorTok{$}\NormalTok{p.q75, }
             \DataTypeTok{xlim=}\KeywordTok{c}\NormalTok{(}\DecValTok{12}\NormalTok{, }\DecValTok{50}\NormalTok{),}
             \DataTypeTok{ylim=}\KeywordTok{c}\NormalTok{(}\DecValTok{100}\NormalTok{,}\OperatorTok{-}\DecValTok{5}\NormalTok{),}
             \DataTypeTok{panel=}\NormalTok{panel.depth_function, }
             \DataTypeTok{prepanel=}\NormalTok{prepanel.depth_function,}
             \DataTypeTok{cf=}\NormalTok{loaf.slab2}\OperatorTok{$}\NormalTok{contributing_fraction,}
             \DataTypeTok{layout=}\KeywordTok{c}\NormalTok{(}\DecValTok{2}\NormalTok{,}\DecValTok{1}\NormalTok{), }\DataTypeTok{scales=}\KeywordTok{list}\NormalTok{(}\DataTypeTok{x=}\KeywordTok{list}\NormalTok{(}\DataTypeTok{alternating=}\DecValTok{1}\NormalTok{))}
\NormalTok{             )}
\end{Highlighting}
\end{Shaded}

\includegraphics{loafercreek_20181221_files/figure-latex/unnamed-chunk-15-1.pdf}

Calculate the maximum clay content from the 1cm slab-wise medians

\begin{Shaded}
\begin{Highlighting}[]
\NormalTok{df <-}\StringTok{ }\KeywordTok{aggregate}\NormalTok{(loaf.slab2}\OperatorTok{$}\NormalTok{p.q50, }\DataTypeTok{by=}\KeywordTok{list}\NormalTok{(loaf.slab2}\OperatorTok{$}\NormalTok{red.shallow), }\DataTypeTok{FUN=}\NormalTok{max, }\DataTypeTok{na.rm =}\NormalTok{ T)}
\KeywordTok{names}\NormalTok{(df) <-}\StringTok{ }\KeywordTok{c}\NormalTok{(}\StringTok{"red.shallow"}\NormalTok{,}\StringTok{"maxclay.q50"}\NormalTok{)}
\NormalTok{df}
\end{Highlighting}
\end{Shaded}

\begin{verbatim}
##   red.shallow maxclay.q50
## 1       FALSE    27.53739
## 2        TRUE    31.81839
\end{verbatim}

Almost a 4.2809964 clay difference there.

\begin{Shaded}
\begin{Highlighting}[]
\CommentTok{#do spatial example.. can we predict where the red/clayey ones are?}
\end{Highlighting}
\end{Shaded}

\begin{Shaded}
\begin{Highlighting}[]
\CommentTok{#bring in the shallow and skeletal and deep data :)}
\end{Highlighting}
\end{Shaded}


\end{document}
